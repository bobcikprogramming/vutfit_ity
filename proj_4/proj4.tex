\documentclass[a4paper, 11pt]{article}
\usepackage[utf8]{inputenc}
\usepackage[top=3cm, text={17cm,24cm}, left=2cm]{geometry}
\usepackage{times}
\usepackage[czech]{babel}
\usepackage[breaklinks]{hyperref}
\usepackage{breakurl}

\title{projekt_4}
\author{xbobci03 }
\date{April 2020}

\begin{document}

\begin{titlepage}
    \begin{center}
        \thispagestyle{empty}
        {\Huge \textsc{Vysoké učení technické v Brně\\[0.4em]}}
        {\huge \textsc{Fakulta informačních technologií}}
        
        \vspace{\stretch{0.382}}
        
        {\LARGE Typografie a~publikování -- 4. projekt\\[0.3em]}
        {\Huge IoT v Průmyslu 4.0}
        \vspace{\stretch{0.618}}
    \end{center}
{\Large 10. dubna 2020 \hfill Pavel Bobčík}
\end{titlepage}

\section{Internet věcí}
Internet věcí, neboli IoT, představuje vzájemnou komunikaci fyzických zařízení pomocí komunikační sítě nebo~internetu. Obdržená data se zpracují a jejich výsledek může být využit k vyhodnocení aktivit jiného zařízení. Jak se uvádí v \cite{Cermak2018}.

Důležitou součástí jsou senzory, jimiž jsou zařízení vybavena. Ty slouží k získávání dat. Není tedy divu, že~je senzorová technologie důležitou součástí internetu věcí. Více lze naleznout na \cite{Fara2019}.

Statistiky z roku 2016 uvádí, že v daném roce bylo připojeno k internetu 1,6 bilionu zařízení. K letošnímu roku se odhaduje až na 50 bilionů kusů, viz \cite{Badurova2019}. Nepřekvapí tedy, že se tento sektor dostává i do průmyslu.

\section{Průmysl 4.0}
\subsection{Původ a myšlenka}
První zmíňka se vyskytla v roce 2011 v Německu. Hlavní myšlenkou je vývoj a inovace technologií orientovaných na oblasti \uv{inteligentních objektů}, jak udává \cite{Fara2019}.
Hlavním konceptem průmyslu 4.0 je tedy chytrá výroba za pomoci IoT, viz \cite{Nagpal2020}.

Jak uvádí článek \cite{Cicvakova2017}, společně s průmyslem 4.0 jde i stále se zvětšující automatizace a optimalizace procesů v oblasti výroby, logistiky a služeb.

Od současného průmyslu se liší především v masivním rozvoji IT a telekomunakcí, viz \cite{Helios2017}.

\subsection{Cíle a výhody}
Cílem je zvýšení flexibility, efektivnosti a souběžné snižování nákladů. Ať už pomocí monitoringu výrobních strojů, správou zásob či spotřeby energie, a další na \cite{Serpanos2018}. Například sledováním stavu zásob může firma ušetřit jak finance za energie, tak i za pronájem prostorů a mzdy zaměstnanců. 

\subsection{Jak toho dosáhnout}
Různé stroje mezi sebou sdílí data v reálném čase. Tyto informace se můžou využít pro daný sektor. Například můžeme dosáhnout úspory energie, jenž průmysl spotřebovává až 35\% světové produkce, jak uvádí \cite{Silva2020}, zefektivněním výrobních činností. Všechna tato data, neboli Big Data, se musí zpracovat, viz \cite{Nagpal2020}.

\subsection{Práce s daty}
Všechna data obdržená pomocí IoT je třeba zpracovat. Často se můžeme setkat s jak cloudovým tak i hybridním řešením. Hybdridní řešení spočívá na tom, že jsou určitá zařízení nebo aplikace s rychlou odezvou řešeny na úrovni závodu, zbytek je prováděn v cloudu, jak se můžeme dočíst na \cite{Varva2020}.

\subsection{Jáka jsou rizika}
Dojde k zmírnění rizika lidské chyby, ale nastanou zde zcela nová. Jedná se o bezpečnostní rizika, jak už tomu bývá (nejen v IT sektoru) zvykem.
Například v logistice dochází k sledování nákladu, a to často dosti podrobně. Nesleduje se pouze samotný náklad, ale například i daná lokomotiva a její vagóny, jak uvádí \cite{Kodym2016}.

Je tedy velice důležité dbát na zabezpečení těchto služeb. A to nejen v průmyslu 4.0, ale u všech zařízení využívajích IoT, jako například chytré domácnosti či náramky a další.

\newpage
\bibliographystyle{czechiso}
\bibliography{proj4}

\end{document}

