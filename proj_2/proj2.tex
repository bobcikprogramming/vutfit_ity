\documentclass[twocolumn, a4paper, 11pt]{article}
\usepackage[utf8]{inputenc}
\usepackage[top=2.5cm, total={18cm,25cm}, left=1.5cm]{geometry}
\usepackage[IL2]{fontenc}
\usepackage[czech]{babel}
\usepackage{times}
\usepackage{amsthm}
\usepackage{amssymb}
\usepackage{mathtools}

\date{}

\begin{document}

\begin{titlepage}
    \begin{center}
        \thispagestyle{empty}
        {\Huge
        \textsc{Fakulta informačních technologií\\[0.4em]
        Vysoké učení technické v Brně}}
        
        \vspace{\stretch{0.382}}
        
        {\LARGE Typografie a~publikování -- 2. projekt\\[0.3em]
        Sazba dokumentů a matematických výrazů}
        \vspace{\stretch{0.618}}
    \end{center}
{\Large 2020 \hfill Pavel Bobčík (xbobci03)}
\end{titlepage}

\newpage

\setcounter{page}{1}
\section*{Úvod}
V této úloze si vyzkoušíme sazbu titulní strany, matematic- kých vzorců, prostředí a dalších textových struktur obvyk- lých pro technicky zaměřené texty (například rovnice (\ref{rovnice2}) nebo Definice \ref{definice2} na straně \pageref{definice2}). Pro vytvoření těchto odkazů používáme příkazy \verb|\label|, \verb|\ref| a \verb|\pageref|.

Na titulní straně je využito sázení nadpisu podle op- tického středu s využitím zlatého řezu. Tento postup byl probírán na přednášce. Dále je použito odřádkování se zadanou relativní velikostí 0.4em a 0.3em.

\section{Matematický text}
Nejprve se podíváme na sázení matematických symbolů a~výrazů v plynulém textu včetně sazby definic a vět s vy- užitím balíku \texttt{amsthm}. Rovněž použijeme poznámku pod čarou s použitím příkazu \verb|\footnote|. Někdy je vhodné použít konstrukci \verb|${}$| nebo \verb|\mbox{}| která říká, že (matematický) text nemá být zalomen. V následující de- finici je nastavena mezera mezi jednotlivými položkami \verb|\item| na 0.05em.

\newtheorem{definice}{Definice}
    \begin{definice}
    \label{definice1}
    \textnormal{Turingův stroj} (TS) je definován jako šestice tvaru 
    $M = (Q,\Sigma,\Gamma,\delta,q_0,q_F)$, kde:

    \begin{itemize}
        \itemsep 0.05em
        \item $Q$ je konečná množina \textnormal{vnitřních (řídicích) stavů,}
        \item $\Sigma$ je konečná množina symbolů nazývaná \textnormal{vstupní
        abeceda, $\Delta \notin  \Sigma$,}
        \item $\Gamma$ je konečná množina symbolů, $\Sigma \subset \Gamma$, $\Delta \in \Gamma$, 
        nazývaná \textnormal{pásková abeceda},
        \item $\delta : (Q \backslash \{q_F\})\times\Gamma\rightarrow Q\times(\Gamma\cup\{L, R\})$, kde $L, R \notin \Gamma$,
        je parciální \textnormal{přechodová funkce,} a
        \item $q_0 \in Q$ je \textnormal{počáteční stav} a $q_f \in Q$ je \textnormal{koncový stav.}
        \end{itemize}
\end{definice}

Symbol $\Delta$ značí tzv. \emph{blank} (prázdný symbol), který se vyskytuje na místech pásky, která nebyla ještě použita.

\emph{Konfigurace pásky} se skládá z nekonečného řetězce, který reprezentuje obsah pásky a pozice hlavy na tomto řetězci. Jedná se o prvek množiny \mbox{\{$\gamma\Delta^\omega \mid \gamma \in \Gamma^\ast$\} $\times \  \mathbb{N}$}\footnote{Pro libovolnou abecedu $\Sigma$ je $\Sigma^\omega$ množina všech \emph{nekonečných} řetězců nad $\Sigma$, tj. nekonečných posloupností symbolů ze $\Sigma$.}. \emph{Konfiguraci pásky} obvykle zapisujeme jako $\Delta xyz\underline{z}x\Delta$... (podtržení značí pozici hlavy). \emph{Konfigurace stroje} je pak dána stavem řízení a konfigurací pásky. Formálně se jedná o prvek množiny $Q\times\{\gamma\Delta^\omega \mid \gamma \in\Gamma^\ast\}\times\mathbb{N}$.

\subsection {Podsekce obsahující větu a odkaz}
\begin{definice}
    \label{definice2}
    \textnormal{Řetězec $w$ nad abecedou $\Sigma$ je přijat TS} $M$
    jestliže $M$ při aktivaci z počáteční konfigurace pásky $\underline{\Delta} w\Delta...$ 
    a počátečního stavu $q_0$ zastaví přechodem do koncového stavu 
    $q_F$, tj. $(q_0, \Delta w\Delta^\omega, 0) \underset{M}{\overset{*}{\vdash}} (q_F , \gamma, n)$ pro
    nějaké $\gamma \in \Gamma^\ast$ a $n \in \mathbb{N}$.
    
    Množinu $L(M) = \{w \mid w$ je přijat TS $M\} \subseteq \Sigma^\ast$~nazý\-váme \textnormal{jazyk přijímaný TS} $M$.  
\end{definice}

Nyní si vyzkoušíme sazbu vět a důkazů opět s použitím
balíku \texttt{amsthm}.

\newtheorem{veta}{Věta}
\begin{veta}
    Třída jazyků, které jsou přijímány TS, odpovídá
    \textnormal{rekurzivně vyčíslitelným jazykům.}
\end{veta}

\begin{proof}
    V důkaze vyjdeme z Definice \ref{definice1} a \ref{definice2}.
\end{proof}

\section{Rovnice}
Složitější matematické formulace sázíme mimo plynulý text. Lze umístit několik výrazů na jeden řádek, ale pak je třeba tyto vhodně oddělit, například příkazem \verb|\quad|.

$$\sqrt[i]{x^3_i} \quad \text{kde $x_i$ je $i$-té sudé číslo}\quad y_i^{2\cdot y_i}\neq y_i^{y_i^{y_i}}$$

V rovnici (\ref{rovnice1}) jsou využity tři typy závorek s různou explicitně definovanou velikostí.

\begin{eqnarray}
    \label{rovnice1}
    x & = & \bigg\{ \Big( \big[a+b\big]\ast c\Big)^d \oplus 1\bigg\}\\
    \label{rovnice2}
    y & = & \lim\limits_{x \to \infty} \frac{\textnormal{\fontfamily{cmr}\selectfont{sin}}^2 x + \textnormal{\fontfamily{cmr}\selectfont{cos}}^2 x}{\frac{1}{\log_{10} x}}
\end{eqnarray}

V této větě vidíme, jak vypadá implicitní vysázení li-
mity $\lim_{n\rightarrow\infty} f(n)$ v normálním odstavci textu. Podobně je to i s dalšími symboly jako $\sum\nolimits_{i=1}^{n}2^i \text{ či } \bigcap_{A\in\mathcal{B}} A$. V~pří\-padě vzorců $\lim\limits_{n \to \infty}f(n) \text{ a } \sum\limits_{i=1}^{n}2^i$ jsme si vynutili méně úspornou sazbu příkazem \verb|\limits|.

\section{Matice}
Pro sázení matic se velmi často používá prostředí \texttt{array} a závorky (\verb|\left|, \verb|\right|).

\begin{equation*}
    \left(
    \begin{array}{ccc}
        a+b & \widehat{\xi + \omega} & \hat{\pi} \\
        \vec{\mathbf{a}} & \overleftrightarrow{AC} & \beta 
    \end{array}
    \right) = 1 \Longleftrightarrow \mathbb{Q} = \mathcal{R}
\end{equation*}
Prostředí \texttt{array} lze úspěšně využít i jinde.

\begin{equation*}
    \binom{n}{k} = 
    \left\{ 
    \begin{array}{c l}
        0 & \mbox{pro $k < 0$ nebo $k > n$}\\
        \frac{n!}{k!(n-k)!} & \mbox{pro $0 \leq k \leq n$}.
    \end{array} 
    \right.
\end{equation*}

\end{document}

